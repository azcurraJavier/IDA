\documentclass[12pt]{report}
\usepackage{graphicx}
\usepackage[spanish]{babel} % Para separar correctamente las palabras
\usepackage[utf8]{inputenc} % Este paquete permite poner acentos y eñes usando codificación utf-8

% Title Page
\title{Análisis de Identificadores para Abstraer conceptos del Dominio del Problema}
\author{Javier Azcurra, Mario Berón\\\\Facultad de Ciencias Físico Matemáticas y Naturales\\Universidad Nacional de San Luis}


\begin{document}
\maketitle
%\tableofcontents %Genera el indice

\begin{abstract}
\end{abstract}

\chapter{Introducción}
\section{Problema}

Uno de los principales problemas al que se ven enfrentados los desarrolladores de 
software es mantener los sistemas en buen funcionamiento \cite{vonmayrhauser1995pcd}. 
Esta tarea es imposible de llevar a cabo de forma manual debido a que consume 
muchos costos y esfuerzo humano. 
Por esta razón, existe una subárea de la Ingeniería de Software que se encuentra 
dedicada al desarrollo de técnicas de inspección y comprensión de software. 
Esta área tiene como principal objetivo que el desarrollador logre un entendimiento 
acabado del software de estudio de forma tal de poder modificarlo disminuyendo en 
lo posible la gran mayoría de costos \cite{BRM10}. 
Esta área se conoce en la jerga de la Ingeniería de Software como: 
\textit{Comprensión de Programas (CP)}.



\section{Solucion}
\begin{itemize}
\item Lograr una mejor comprensión de programas escritos en java a través del análisis de los identificadores encontrados en el código fuente de programas escritos con java.
\end{itemize}



\bibliographystyle{plain}
\bibliography{biblo.bib}

\end{document}