\documentclass[12pt]{report}
\usepackage{graphicx}
\usepackage[spanish]{babel} % Para separar correctamente las palabras
\usepackage[utf8]{inputenc} % Este paquete permite poner acentos y eñes usando codificación utf-8

% Title Page
\title{Análisis de Identificadores para Abstraer conceptos del Dominio del Problema}
\author{Javier Azcurra\\\\Trabajo Final de Licenciatura en Cs. de la Computación\\\\\\Facultad de Ciencias Físico Matemáticas y Naturales\\Universidad Nacional de San Luis}

%\centering
%\includegraphics[scale= 0.18]{./unsl_logo.JPG}

\begin{document}
\maketitle

%\renewcommand{\abstractname}{Agradecimientos}
%\begin{abstract}
% Gracias!
%\end{abstract}

%\tableofcontents %Genera el indice

\begin{abstract}
Las demandas actuales en el desarrollo del software implican una evolución y mantenimiento constante del mismo con el menor costo de tiempo y de recursos. Pensar en estrategias que faciliten las tediosas tareas que diariamente conllevan al crecimiento de los sistemas nos da incapie a iniciarnos en la investigación de herramientas automatizadas que posibiliten el reemplazo del esfuerzo manual que realizan los ingenieros de software a la hora de interpretar un programa.

Estudios indican que los identificadores abundan en los códigos de programas y poseen información relevante detrás de sus abreviaturas por lo tanto no debe ser pasados por alto su análisis a la hora de elaborar herramientas automatizadas de interpretación de códigos.
El correspondiente trabajo habla de técnicas basadas en el análisis de identificadores en códigos escritos en lenguaje JAVA\texttrademark .
Las técnicas de análisis constan de dos etapas la primera consiste en la división de las distintas abreviaturas que componen un identificador y la segunda etapa se encarga en la expansión ``semántica'' de cada abreviatura capturada en la etapa previa, es decir, expandir las abreviaciones en palabras completas. Para lograr esta expansión se utilizan distintas fuentes semánticas comentarios, literales y documentación o bien fuera del sistema basándose en diccionarios. Algunas estrategias de análisis de identificadores se implementaron en la herramienta \textit{Identifier Analyzer} (IDA) donde se usaron técnicas de compilación para extraer la información estática y analizar los identificadores con el objeto de poder comparar el desempeño de las estrategias y arribar a conclusiones.




\end{abstract}

\chapter{Introducción}
\section{Problema}

La ingeniería de software contiene tres temáticas muy importantes en desarrollo de los sistemas: el \textit{mantenimiento del software}, \textit{la evolución del software} y la \textit{migración del software}.

La etapa de mantenimiento es importante en el desarrollo de los sistemas porque los mismos están sujetos a cambios y a una permanente evolución\cite{PFT02}.
Es común también que por la constate actualización de los sistemas operativos, los motores de base de datos y demás sistemas externos que interactúan con el software desarrollado entren en conflicto, por eso en la fase de mantenimiento también se debe ir actualizando los distintos componentes del producto para una mejor compatibilidad.\cite{RSPMGH02}. Por lo antedicho entre otros diversas razones indican que el mantenimiento del software consume mucho esfuerzo y dinero. Es necesario pensar en estrategias de automatización que puedan ser aplicadas en fases del mantenimiento del software que ayuden a reducir estos costos, para lograrlo se requiere una comprensión del objeto que se va a modificar antes de realizar algún cambio que sea de utilidad.

Por otro lado la evolución del software se atribuye al crecimiento de los sistemas, es decir, tomar una versión operativa y generar una nueva versión ampliada.
Los sistemas complejos evolucionan con el tiempo, los nuevos usuarios y requisitos durante el desarrollo del mismo, causan que el producto final posiblemente no haya sido el que se planteó en un comienzo. 
Generalmente los ingenieros del software utilizan modelos de procesos conocidos que se diseñaron de antemano para adaptarse a un producto que evoluciona con el tiempo.\cite{RSPMGH02}. Obligadamente se requiere una previa interpretación del sistema.


La migración del software es una tarea fundamental y compleja dentro del mantenimiento del software. El objetivo es convertir un viejo sistema dentro de una nueva tecnología sin cambiar la funcionalidad del mismo, lograr esto es costoso \cite{WHAFVR11}. Por eso es fundamental lograr una comprensión del sistema antiguo, esto implica que se debe elevar el código antiguo a un nivel más alto de abstracción tomando características principales \cite{MMFAF07}.

%WICC:
Todos los problemas a los que se enfrentan los desarrolladores de software el primordial es el de mantener los sistemas en buen funcionamiento \cite{VMAVA95}. 
Esta tarea es imposible de llevar a cabo de forma manual debido a que consume 
muchos costos y esfuerzo humano. 
Por esta razón, existe una subárea de la Ingeniería de Software que se  
dedica al desarrollo de técnicas de inspección y comprensión de software. 
Esta área tiene como principal objetivo que el desarrollador logre un entendimiento 
acabado del software de estudio de forma tal de poder modificarlo disminuyendo en lo posible la gran mayoría de costos \cite{BRM10}. 
El área mencionada se conoce en la jerga de la Ingeniería de Software como: 
\textit{Comprensión de Programas (CP)}.

Uno de los principales desafíos en CP consiste en relacionar 
dos dominios muy importantes. 
El primero, el Dominio del Problema, hace referencia a la salida producida por el sistema de estudio. El segundo, el Dominio del Programa, se refiere a las componentes de software utilizadas para producir dicha salida.

Los caminos que conducen a facilitar la comprensión de software el mas apropiado consiste 
en el uso/creación de Herramientas de Comprensión. 
Una Herramienta de Compresión presenta diferentes perspectivas del software que posibilitan que el ingeniero pueda percibir el funcionamiento del sistema. 
Para construir herramientas de comprensión, se deben tener en cuenta tres pilares importantes, ellos son: \textit{Interconexión de Dominios}, 
\textit{Visualización de Software} y 
\textit{Extracción de la Información} \cite{STOREY99,BROOK82}.

La \textit{visualización del software} es una característica importante en la comprensión de programas, básicamente provee una o varias representaciones visuales de algún sistema particular \cite{BRM10}.
Dichas vistas, cuando están bien elaboradas, permiten analizar y percibir la información extraída desde un programa con mayor facilidad.

Por \textit{Extracción de la Información} se entiende el uso/desarrollo de técnicas que 
permitan extraer información desde el sistema de estudio. 
Esta información puede ser: Estática o Dinámica, dependiendo de las necesidades del 
ingeniero de software o del equipo de trabajo.

Para la extracción de la información estática se utilizan técnicas de compilación tradicionales, que se encargan de recuperar información de cada componente del sistema. Todas las actividades que forman parte de esta tarea se realizan desde el código fuente sin ejecutar el sistema. Generalmente, en este tipo de trabajos se construye un analizador sintáctico con las acciones semánticas necesarias para extraer la información requerida.
Por otro lado la extracción dinámica de información del sistema se obtiene  
aplicando técnicas de instrumentación de código, estas técnicas consisten en insertar sentencias dentro del código fuente del sistema con el fin de recuperar las partes del programa que se utilizaron para 
producir la salida. 
La principal diferencia que radica entre ambas técnicas es que las dinámicas requieren que el sistema se ejecute, mientras que las estáticas esto no es necesario.

\begin{itemize}
\item Lograr una mejor comprensión de programas escritos en java a través del análisis de los identificadores encontrados en el código fuente de programas escritos con java.
\end{itemize}


\section{Solución}

\begin{itemize}
\item Aplicar distintas técnicas de análisis de identificadores para facilitar la comprensión de programas.
\end{itemize}

Una de las vías mas sencillas de comprender grandes sistemas es relacionar el Dominio del Problema con el Dominio del Programa es por ello la necesidad de utilizar los conceptos basados en la \textit{Interconexión de Dominios} \cite{BRM10}.
Esta relación entre ambos dominios es compleja y se puede aproximar primero construyendo una representación de ambos dominios y luego llevar a cabo una estrategia de vinculación entre ambas representaciones.

\pagebreak %salto de pagina
\section{Contribución}
La línea de investigación descrita en este trabajo final se encuentra enmarcada en el 
contexto del proyecto: \textit{Ingeniería del Software: Conceptos Métodos Técnicas y 
Herramientas en un Contexto de Ingeniería de Software en Evolución} de la Universidad 
Nacional de San Luis. 
Dicho proyecto, es reconocido por el programa de incentivos y es la continuación de 
diferentes proyectos de investigación de gran éxito a nivel nacional e internacional.

También se forma parte del proyecto bilateral entre la Universidade do Minho (Portugal)
 y la Universidad Nacional de San Luis (Argentina) denominado \textit{Quixote: Desarrollo de 
Modelos del Dominio del Problema para Inter-relacionar las Vistas Comportamental y 
Operacional de Sistemas de Software}. Quixote\footnote[1]{http://www3.di.uminho.pt/~gepl/} fue aprobado por el 
Ministerio de Ciencia, Tecnología e Innovación Productiva de la Nación 
(MINCyT) y la Fundação para a Ciência e Tecnología (FCT) de Portugal. 
Ambos entes soportan económicamente la realización de diferentes misiones de investigación desde Argentina a Portugal y viceversa.

\section{Organización de Trabajo Final}
En este trabajo final se presenta una línea de investigación que se centra en el estudio, 
creación e implementación de técnicas de extracción de la información estática desde 
los sistemas de software. 
Esta información puede ser estrictamente relacionada con el código del programa, 
o bien con la información informal provista por los programadores a través de 
comentarios, literales y documentación. El trabajo esta organizado de la siguiente manera. El capítulo 2 define conceptos teóricos relacionados a la comprensión de programas, cuales son las necesidades que implica su estudio y que soluciones presenta en la vida del desarrollo del software. El capítulo 3 habla de técnicas de análisis de identificadores y que fuentes semánticas son utilizadas para su interpretación. El capítulo 4 trata sobre la herramienta \textit{Identifier Analyzer} (IDA) que implementa algunas técnicas de análisis de identificadores sobre códigos escritos en JAVA\texttrademark y algunos casos de estudio. El capítulo 5 menciona las conclusiones elaboradas y posibles trabajos futuros.

\bibliographystyle{plain}%{alpha}
\bibliography{biblo.bib}

\end{document}