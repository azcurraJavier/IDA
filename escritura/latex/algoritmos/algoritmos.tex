\documentclass[a4paper,12pt]{report}

\usepackage{graphicx}
\usepackage[spanish, es-tabla]{babel} % Para separar correctamente las palabras, es-tabla: para que nombre a la tabla como tal y no cuadro.

\usepackage[utf8]{inputenc} % Este paquete permite poner acentos y eñes usando codificación utf-8
\usepackage{setspace}
\onehalfspace %para espacio y medio

\usepackage{amsmath}%formulas y funciones mat

\newcommand{\at}{\makeatletter @\makeatother}%comando define la arroba porque no forma parte de latex

%=====formato para escribir pseudo-algortimo.============
%algorithm2e: para que ande 'sudo apt-get install texlive-science'
%aqui se encuentra instalado:
%/usr/share/texmf-texlive/tex/latex/algorithm2e/algorithm2e.sty
\usepackage[spanish,ruled,vlined,onelanguage]{algorithm2e} 
%ruled, vlined para formato de lineas.  
%spanish setea idioma español
%onelanguage español las palabras reservadas
%===================================================


\begin{document}

\begin{algorithm}[H]%H fuerza que se genere aqui
\LinesNumbered%enumera las lineas
\SetKwInOut{Input}{Entrada}\SetKwInOut{Output}{Salida}
\SetKwInOut{Global}{Var Global}
\Global{\textit{\textbf{ispellList}} \tcp{\textit{Palabras de ispell + Diccionario}}}
\Global{\textit{\textbf{abrevList}} \tcp{\textit{Abreviaciones conocidas}}}
\Global{\textit{\textbf{stopList}} \tcp{\textit{Palabras Excluyentes}}}
\Input{\textit{\textbf{idHarword}} \tcp{\textit{identificador a dividir}}}
\Output{\texttt{softwordDiv} \tcp{\textit{id separado con espacios}}}

\BlankLine
\texttt{softwordDiv} $\leftarrow$ “”

\texttt{softwordDiv} $\leftarrow$ dividirCaracteresEspecialesDigitos(\textit{\textbf{idHarword}})

\texttt{softwordDiv} $\leftarrow$ dividirCamelCase(\texttt{softwordDiv})

\BlankLine
\ForAll{$($\textit{\textbf{s}} $|$ \textit{\textbf{s}} es un substring separado por ` ' en \texttt{softwordDiv}$)$}{
\BlankLine

\If{$($\textbf{s} no pertenece a $($\textit{\textbf{stopList}} $\cup$ \textit{\textbf{abrevList}} $\cup$ \textbf{ispellList} $))$}{
\BlankLine
resultadoPrefijo $\leftarrow$ buscarPrefijo(\textit{\textbf{s}},“”)

resultadoSufijo $\leftarrow$ buscarSufijo(\textit{\textbf{s}},“”)
\BlankLine
\tcp{\textit{Se elije la división que mayor particiones hizo.}}
\textit{\textbf{s}} $\leftarrow$ maxDivisión(resultadoPrefijo,resultadoSufijo)

}
}
\BlankLine
\Return \texttt{softwordDiv} \tcp{\textit{Retorna el id dividido por ` '}}

\caption{División Greedy\label{GSA}}
\end{algorithm}

\begin{function}[H]
\LinesNumbered%enumera las lineas
\SetKwInOut{Input}{Entrada}\SetKwInOut{Output}{Salida}
\Input{\textit{\textbf{s}} \tcp{\textit{Abreviaturas a dividir}}}
\Output{\textit{\textbf{abrevSeparada}} \tcp{\textit{Abreviaturas separadas}}}
\BlankLine
\tcp{\textit{Punto de parada de la recursión.}}
\If{$($length$($\textit{\textbf{s}}$)$ $=$ 0$)$}{
\BlankLine
\Return \textit{\textbf{abrevSeparada}}
}
\BlankLine
\If{$($\textbf{s} pertenece a $($\textit{\textbf{stopList}} $\cup$ \textit{\textbf{abrevList}} $\cup$ \textbf{ispellList} $))$}{
\BlankLine
%\tcp{\textit{Se separa con espacio y se ejecuta la función con el resto.}}
\BlankLine
\Return (\textit{\textbf{s}} $+$ ` ' $+$ buscarPrefjjo(\textit{\textbf{abrevSeparada}},“”) )
}
\BlankLine
\tcp{\textit{Se extrae y se guarda el último caracter de s.}}

\textit{\textbf{abrevSeparada}} $\leftarrow$ \textit{\textbf{s}}[length(\textit{\textbf{s}}) - 1] $+$ \textit{\textbf{abrevSeparada}}
\BlankLine

\tcp{\textit{Llamar nuevamente a la función sin el último caracter.}}
\BlankLine
\textit{\textbf{s}} $\leftarrow$  \textit{\textbf{s}}[0 , length(\textit{\textbf{s}}) - 1]

\Return buscarPrefjjo(\textit{\textbf{s}},\textit{\textbf{abrevSeparada}})

\caption{buscarPrefjjo()}
\end{function}

\begin{function}[H]
\LinesNumbered%enumera las lineas
\SetKwInOut{Input}{Entrada}\SetKwInOut{Output}{Salida}
\Input{\textit{\textbf{s}} \tcp{\textit{Abreviaturas a dividir}}}
\Output{\textit{\textbf{abrevSeparada}} \tcp{\textit{Abreviaturas separadas}}}
\BlankLine
\tcp{\textit{Punto de parada de la recursión.}}
\If{$($length$($\textit{\textbf{s}}$)$ $=$ 0$)$}{
\BlankLine
\Return \textit{\textbf{abrevSeparada}}
}
\BlankLine
\If{$($\textbf{s} pertenece a $($\textit{\textbf{stopList}} $\cup$ \textit{\textbf{abrevList}} $\cup$ \textbf{ispellList} $))$}{
\BlankLine
%\tcp{\textit{Se separa con espacio y se ejecuta la función con el resto.}}
\BlankLine
\Return (buscarSufijo(\textit{\textbf{abrevSeparada}},“”) $+$ ` ' $+$ \textit{\textbf{s}})
}
\BlankLine
\tcp{\textit{Se extrae y se guarda el primer caracter de s.}}

\textit{\textbf{abrevSeparada}} $\leftarrow$ \textit{\textbf{abrevSeparada}} $+$ \textit{\textbf{s}}[0]
\BlankLine
\tcp{\textit{Llamar nuevamente a la función sin el primer caracter.}}
\BlankLine
\textit{\textbf{s}} $\leftarrow$ \textit{\textbf{s}}[1 , length(\textit{\textbf{s}})]

\Return buscarSufijo(\textit{\textbf{s}},\textit{\textbf{abrevSeparada}})

\caption{buscarSufijo()}
\end{function}

%=================

\begin{algorithm}
\LinesNumbered%enumera las lineas
\SetKwInOut{Input}{Entrada}\SetKwInOut{Output}{Salida}
\Input{\textbf{\textit{token}} \tcp{\textit{token a dividir}}}
\Output{\textbf{\textit{tokenSep}} \tcp{\textit{token separado con espacios}}}
\BlankLine
\textbf{\textit{token}} $\leftarrow $ dividirCaracteresEspecialesDigitos(\textbf{\textit{token}})

\textbf{\textit{token}} $\leftarrow $ dividirMinusSeguidoMayus(\textbf{\textit{token}})

\textbf{\textit{tokenSep}} $\leftarrow $ “”
\BlankLine
\ForAll{$($\textbf{\textit{s}} $|$ \textbf{\textit{s}} es un substring  separado por ` ' en \textbf{\textit{token}}$)$}{
\BlankLine

\If{$($ $\exists \{ i|esMayus(\textbf{s}[i])  \land  esMinus(\textbf{s}[i+1]) \}$ $)$}{
\BlankLine
$n \leftarrow$ length(\textbf{\textit{s}}) $-$ 1
\BlankLine
\tcp{\textit{se determina con la función score si es del tipo camelcase u otra alternativa}} 
\BlankLine
$scoreCamel \leftarrow$ score(\textbf{\textit{s}}[i,n])   

$scoreAlter \leftarrow$ score(\textbf{\textit{s}}[i+1,n])  
\BlankLine
\If{$(scoreCamel > \sqrt{scoreAlter})$}{
\BlankLine
\If{$($i $>$ 0$)$}{
\BlankLine
\textbf{\textit{s}} $\leftarrow$ \textbf{\textit{s}}[0,i $-$ 1] + ` ' + \textbf{\textit{s}}[i,n] \tcp{\textit{GP Sstate}}
}

}\Else{
\BlankLine
\textbf{\textit{s}} $\leftarrow$ \textbf{\textit{s}}[0,i] $+$ ` ' $+$ \textbf{\textit{s}}[i $+$ 1,n]  \tcp{\textit{GPS state}}
}

}
\BlankLine
\textbf{\textit{tokenSep}} $\leftarrow$ \textbf{\textit{tokenSep}} $+$ ` ' $+$\textbf{\textit{s}}

}
\BlankLine
\textbf{\textit{token}} $\leftarrow$ \textbf{\textit{tokenSep}}

\textbf{\textit{tokenSep}} $\leftarrow$ ` '
\BlankLine
\ForAll{$($\textbf{\textit{s}} $|$ \textbf{\textit{s}} es un substring separado por ` ' en \textbf{token}$)$}{
\BlankLine
\textbf{\textit{tokenSep}} $\leftarrow$ \textbf{\textit{tokenSep}} $+$ ` ' $+$ \textbf{divisiónSoftWord}(\textbf{\textit{s}},score(\textbf{\textit{s}}))
}
\BlankLine
\Return \textbf{\textit{tokenSep}}

\caption{divisiónHardWord \label{AHW}}
\end{algorithm}

%%=================

\begin{algorithm}
\LinesNumbered%enumera las lineas
\SetKwInOut{Input}{Entrada}\SetKwInOut{Output}{Salida}
\Input{\textbf{\textit{s}} \tcp{\textit{softword string}}}
\Input{\textbf{\textit{score$_{sd}$}} \tcp{\textit{puntaje de s sin dividir}}}
\Output{\textbf{\textit{tokenSep}} \tcp{\textit{token separado con espacios}}}
\BlankLine%espacio en blanco

\textbf{\textit{tokenSep}} $\leftarrow$ \textbf{\textit{s}}, n $\leftarrow$ length(\textbf{\textit{s}}) - 1

i $\leftarrow$ 0, \textbf{\textit{maxScore}} $\leftarrow -$ 1
\BlankLine

\While{$($i $<$ n$)$}{
\BlankLine
\textbf{\textit{score$_{izq}$}} $\leftarrow$ score(\textbf{\textit{s}}[0,i])

\textbf{\textit{score$_{der}$}} $\leftarrow$ score(\textbf{\textit{s}}[i+1,n])
\BlankLine
\textbf{\textit{preSuf}} $\leftarrow$ esPrefijo(\textbf{\textit{s}}[0,i]) $\vee$ esSufijo(\textbf{\textit{s}}[i+1,n])
\BlankLine
\textbf{\textit{split$_{izq}$}} $\leftarrow$  $\sqrt{score_{izq}} >$ max(score(\textbf{\textit{s}}),\textbf{\textit{score$_{sd}$}})

\textbf{\textit{split$_{der}$}} $\leftarrow$  $\sqrt{score_{der}} >$ max(score(\textbf{\textit{s}}),\textbf{\textit{score$_{sd}$}})
\BlankLine

\If{$($!\textbf{\textit{presuf}} $\land$ \textbf{\textit{split$_{izq}$}} $\land$ \textbf{\textit{split$_{der}$}}$)$}{
\BlankLine
\If{$(($\textbf{\textit{split$_{izq}$}} $+$ \textbf{\textit{split$_{der}$}}$)$ $>$ \textbf{maxScore}$)$}{
\BlankLine
\textbf{\textit{maxScore}} $\leftarrow$ (\textbf{\textit{split$_{izq}$}}$+$\textbf{\textit{split$_{der}$}})

\textbf{\textit{tokenSep}} $\leftarrow$ \textbf{\textit{s}}[0,i] + ` ' + \textbf{\textit{s}}[i+1,n]
}

}\ElseIf{$(!$\textbf{\textit{presuf}} $\land$ \textbf{\textit{split$_{izq}$}}$)$}{
\BlankLine
\textbf{\textit{temp}} $\leftarrow$ \textbf{divisiónSoftWord}(\textbf{\textit{s}}[i+1,n],\textbf{\textit{score$_{sd}$}})

\If{$($\textbf{\textit{temp}} se dividió?$)$}{
\textbf{\textit{tokenSep}} $\leftarrow$ \textbf{\textit{s}}[0,i] + ` ' + \textbf{\textit{temp}}
}

}
\BlankLine
i $\leftarrow$ i+1
}
\BlankLine
\Return \textbf{\textit{tokenSep}}

\caption{divisiónSoftWord \label{ASW}}
\end{algorithm}

%%=================

\begin{algorithm}
\LinesNumbered%enumera las lineas
\SetKwInOut{Input}{Entrada}\SetKwInOut{Output}{Salida}
\Input{\textit{\textbf{abrev}} \tcp{\textit{Abreviatura a expandir}}}
\Input{\textit{\textbf{wordList}} \tcp{\textit{Palabras extraídas del código}}}
\Input{\textit{\textbf{phraseList}} \tcp{\textit{Frases extraídas del código}}}
\Input{\textit{\textbf{stopList}} \tcp{\textit{Palabras Excluyentes}}}
\Input{\textit{\textbf{dicc}} \tcp{\textit{\textit{Diccionario en Inglés}}}}
\Output{\textbf{únicaExpansión} \tcp{\textit{Abreviatura expandida, o null}}}
\BlankLine

\If{$($\textit{\textbf{abrev}} pertenece \textit{\textbf{stopList}}$)$}{
\Return null
}
\BlankLine
listaExpansión $\leftarrow$ [ ]
\BlankLine
\tcp{\textit{Buscar coincidencia de acrónimo.}}

\ForAll{$($\textit{\textbf{phrase}} $|$ \textit{\textbf{phrase}} es una frase en \textit{\textbf{phraseList}}$)$}{
\BlankLine

\If{$($\textit{\textbf{abrev}} es un acrónimo de \textit{\textbf{phrase}}$)$}{
listaExpansión.add(\textit{\textbf{phrase}}) 
}
}

\BlankLine
\tcp{\textit{Buscar abreviatura común.}}

\ForAll{$($\textit{\textbf{word}} $|$ \textit{\textbf{word}} es una palabra en  \textit{\textbf{wordList}} $)$}{
\BlankLine
\If{ $($\textit{\textbf{abrev}} es una abreviatura de \textit{\textbf{word}}$)$}{
listaExpansión.add(\textit{\textbf{word}}) 
}

}

\BlankLine
\tcp{\textit{Si no hay éxito, buscar en el diccionario.}}

\If{$($isEmpty$($listaExpansión$))$}{
\BlankLine
listaCandidatos $\leftarrow$ buscarDiccionario(\textit{\textbf{abrev}},\textit{\textbf{dicc}})
listaExpansión.add(listaCandidatos)
}


\BlankLine
\textbf{únicaExpansión} $\leftarrow$ null
\BlankLine
\tcp{\textit{Debe haber un solo resultado, sino no retorna nada.}}

\If{$($length$($listaExpansión$)$ $=$ 1$)$}{
\BlankLine
\textbf{únicaExpansión} $\leftarrow$ listaExpansión[0]
}
\BlankLine
\Return \textbf{únicaExpansión} 

\caption{Expansión Básica \label{BEA}}
\end{algorithm}





%%=================

\begin{algorithm}
\LinesNumbered%enumera las lineas
\SetKwInOut{Input}{Entrada}\SetKwInOut{Output}{Salida}
\Input{\textit{\textbf{pa}} \tcp{\textit{Palabra Abreviada}}}
\Input{\textit{\textbf{patrón}} \tcp{\textit{Expresión regular}}}
\Input{Cuerpo y Comentarios del Método}
\Input{Comentarios de la Clase}
\Output{Palabras largas candidatas, o null si no hay}
\BlankLine
\tcp{\textit{Las expresiones regulares están entre comillas}}
\BlankLine

\If{ $($\textbf{patrón} prefijo $\vee$ \textit{\textbf{pa}} coincide “[a-z][\^{}aeiou]$+$” $\vee$ length(\textit{\textbf{pa}}) $>$ 3$)$ $\land$ $($\textit{\textbf{pa}} no coincide con “[a-z][aeiou][aeiou]$+$”$)$}{


\BlankLine
\tcp{\textit{Si alguna de las siguientes búsquedas encuentra un único resultado, el algoritmo lo retorna finalizando la ejecución}}
\BlankLine
Buscar en Comentarios JavaDoc con “\textsf{\at param} \textit{\textbf{pa patrón}}”
\BlankLine
Buscar en Nombres de Tipos y la correspondiente Variable declarada con “\textit{\textbf{patrón pa}}”
\BlankLine
Buscar en el Nombre del Método con “\textit{\textbf{patrón}}”
\BlankLine
Buscar en las Sentencias con “\textit{\textbf{patrón pa}}” y “\textit{\textbf{pa patrón}}”
\BlankLine
\If{$($length(\textit{\textbf{pa}}) $\neq$ 2$)$}{
\BlankLine
Buscar en palabras del Método con “\textit{\textbf{patrón}}”
\BlankLine
Buscar en palabras que están en los Comentarios del Método con “\textit{\textbf{patrón}}”

}

\If{$($length(\textit{\textbf{pa}}) $>$ 1$)$ $\land$ $($\textit{\textbf{patrón}} prefijo$)$}{
\BlankLine
\tcp{\textit{Solo se busca con patrones prefijos}}

Buscar en palabras que están en los Comentarios de la Clase con “\textit{\textbf{patrón}}”
}
}

\caption{Búsqueda por Palabras Singulares \label{BPS}}
\end{algorithm}


%%=================

\begin{algorithm}
\LinesNumbered%enumera las lineas
\SetKwInOut{Input}{Entrada}\SetKwInOut{Output}{Salida}
\Input{\textit{\textbf{pa}} \tcp{\textit{Palabra Abreviada}}}
\Input{\textit{\textbf{patrón}} \tcp{\textit{Expresión regular}}}
\Input{Cuerpo y Comentarios del Método}
\Input{Comentarios de la Clase}
\Output{Palabras largas candidatas, o null si no hay}
\BlankLine
\tcp{\textit{Las expresiones regulares están entre comillas}}
\BlankLine

\If{$($\textit{\textbf{patrón}} acrónimo $\vee$ length(\textit{\textbf{pa}}) $>$ 3$)$}{

\BlankLine
\tcp{\textit{Si alguna de las siguientes búsquedas encuentra un único resultado, el algoritmo lo retorna finalizando la ejecución}}
\BlankLine
Buscar en Comentarios JavaDoc con “\textsf{\at param} \textit{\textbf{pa patrón}}”
\BlankLine
Buscar en Nombres de Tipos y la correspondiente Variable declarada con “\textit{\textbf{patrón pa}}”
\BlankLine
Buscar en el Nombre del Método con “\textit{\textbf{patrón}}”
\BlankLine
Buscar en todos los ids (y sus tipos) dentro del Método con “\textit{\textbf{patrón}}”
\BlankLine
Buscar en Literales String con “\textit{\textbf{patrón}}”
\BlankLine
\tcp{\textit{En este punto se buscó en todos los lugares posibles dentro del método}}
\BlankLine
Buscar en palabras que están en los Comentarios del Método con “\textit{\textbf{patrón}}”
\BlankLine
\If{$($\textit{\textbf{patrón}} acrónimo$)$}{
\BlankLine
\tcp{\textit{Solo se busca con patrones Acrónimos}}
Buscar en palabras que están en los Comentarios de la Clase con “\textit{\textbf{patrón}}”
}
}

\caption{Búsqueda por Multi Palabras \label{BMP}}
\end{algorithm}

\end{document}


