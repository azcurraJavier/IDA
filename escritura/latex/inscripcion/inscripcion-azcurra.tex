\documentclass[a4paper,12pt]{report}
\usepackage[spanish, es-tabla]{babel}
\usepackage[utf8]{inputenc} 
\usepackage[margin=1.1in]{geometry}

\usepackage{setspace}
\onehalfspace %para espacio y medio
\setlength\parindent{0pt}%elimina sangria

\begin{document}

{\Large \textbf{Título:}} Análisis de Identificadores para Abstraer conceptos del Dominio del Problema.
\vskip0.5cm
\textbf{Autor:} Javier Azcurra Marilungo.
\vskip0.5cm
\textbf{Dirección:}
\begin{itemize}
\itemsep0em%reduce espacio
\item \textbf{Director:} Dr. Mario Marcelo Berón.

\item \textbf{Co-Director:} Dr. Germán Antonio Montejano.
\end{itemize}

\textbf{Objetivos:} Los objetivos principales de este trabajo final de Licenciatura son:
\begin{itemize}
\itemsep0em%reduce espacio
\item Extraer identificadores de programas escritos en lenguaje JAVA.

\item Analizar los identificadores extraídos.

\item Construir en JAVA una herramienta que implemente los ítems anteriores.

\item Evidenciar las posibilidades de relacionar la salida de un programa con los elementos involucrados que producen dicha salida.
\end{itemize}

\textbf{Antecedentes:}
\vskip0.5cm
%\hspace{0.5cm}  %Sangria
\hspace{0.5cm} La Comprensión de Programas (CP) \cite{BRM10,MPMR07,MBPHRU10,MAS05} es un área de la Ingeniería de Software cuyo objetivo 
principal es desa\-rrollar métodos, técnicas y herramientas que faciliten al programador 
el entendimiento de las funcionalidades de los sistemas de software.
Una forma de alcanzar este objetivo consiste en relacionar el Domino del Problema, 
es decir la salida del sistema, con el dominio del programa, o sea 
con las partes del programa utilizadas para generar la salida del sistema \cite{VMAVA95,MPOB03,BROOK82}.
La construcción de esta relación representa el principal desafío en el contexto de la 
CP. Una solución posible al desafío previamente mencionado consiste en construir 
una representación para cada dominio y luego vincular ambas representaciones.
La representación de ambos dominios se construye en base a la información, 
estática y dinámica, que se extrae de los mismos. En la actualidad se conocen distintas técnicas y herramientas que llevan a cabo esta tarea \cite{AHUL06,THBE99}.
 
\hspace{0.5cm}La información dinámica requiere que el sistema sea modificado sin cambiar su 
semántica y luego ejecutado.



La información estática se extrae desde el código fuente del sistema usando 
técnicas de compilación.

\textbf{Metodología:}
\vskip0.5cm



\textbf{Plan y Cronograma de tareas:}
\vskip0.5cm


\begin{enumerate}
\itemsep0em%reduce espacio
\item Estudiar las técnicas, herramientas y conceptos relacionados con la comprensión de programas: esto permite conocer todos aquellos aspectos vinculados con la comprensión de sistemas;
\end{enumerate}

\textbf{Recursos:}
\vskip0.5cm
\hspace{0.5cm}El correspondiente trabajo final se llevará a cabo en la Universidad Nacional de San Luis, en el Área Programación y Metodologías de Desarrollo de Software enmarcado en los siguientes proyectos:

\begin{itemize}
\itemsep0em%reduce espacio
\item \textit{“Ingeniería del Software: Conceptos Métodos Técnicas y 
Herramientas en un Contexto de Ingeniería de Software en Evolución”} de la Universidad 
Nacional de San Luis. 
Dicho proyecto, es reconocido por el programa de incentivos y es la continuación de 
diferentes proyectos de investigación de gran éxito a nivel nacional e internacional.

\item “Quixote - Development of Problem Domain Models to Interconnect the Behavioral and Operational Views to Aid in Software Systems Comprehension” (código de proyecto: PO/09/38).  Quixote es un proyecto bilateral entre la Universidade do Minho (Portugal) y la Universidad Nacional de San Luis (Argentina). Dicho proyecto fue aprobado por el Ministerio de Ciencia, Tecnología e Innovación Productiva de la Nación (MinCyT)\footnote[1]{www.mincyt.gov.ar/}, y la Fundação para a Ciência e Tecnología (FCT)\footnote[2]{www.fct.mctes.pt/} de Portugal. 

\end{itemize}

\hspace{0.5cm}Ambos entes soportan económicamente la realización de diferentes misiones de investigación desde Argentina a Portugal y viceversa, como así también la presentación y publicación de artículos en diferentes congresos nacionales e internacionales.\\

\hspace{0.5cm}Por otro lado, el equipamiento necesario es una PC con el sistema Linux, impresora y papel para realizar las impresiones del informe. También se hará uso de internet, de material bibliográfico provisto por la biblioteca “Antonio Esteban Agüero”; y  de material provisto por las librerías digitales a las que se tiene acceso desde la Universidad.



%========================================
\renewcommand{\bibname}{\normalsize \textbf{Bibliografía}}

\bibliographystyle{plain}%{alpha}
\bibliography{biblo}
\nocite{*}%para que imprima toda la referencias sin necesidad de citar

{\textbf{\\\\Firma de alumno y asesores}}

\end{document}